\documentclass[10pt,conference,compsocconf]{IEEEtran}

\usepackage{hyperref}
\usepackage{graphicx}
\usepackage{xcolor}
\usepackage{blindtext, amsmath, comment, subfig, epsfig }
\usepackage{grffile}
\usepackage{caption}
\usepackage{subcaption}
\usepackage{algorithmic}
\usepackage[utf8]{inputenc}


\title{CS523 Project 1 Report}
\author{Wicky Simon, Nunes Silva Daniel Filipe}
\date{February 2020}

\begin{document}

\maketitle

\begin{abstract}
This project consists in the implementation a $N$-party multiparty
computations system. Unlike a traditional approach, this aims to compute
the result of a given circuit using users secret inputs without requiring
them to reveal them explicitely. To achieve it, our Go program must be runned
by each of the users who provide their secret values and the ciruit they
want to compute together. Then, they share their additive secret sharings
across the network they are linked with, parse the circuit, generate Beaver
triplets when necessary and evaluate the result before retrieving it.
We analyze the scenarios and the consequenting tradeoffs in which the users
have access to a trusted third-party and the one in which they have not. We
make use of the Lattigo library for the cryptographic operations and the
algebraic structure implementations it provides.

Please report your design, implementation details, findings of the first project in this report. \\
You can add references if necessary \cite{article}. \\
THE REPORT SHOULD NOT EXCEED 3 PAGES.
\end{abstract}
\section{Introduction}
The aim of this project is to design, implement and assess two MPC engines using
the Go programming language. First, two weeks are dedicated to the understanding
of the general architecture, how to use it and how to tweak it to perform
computations in a privacy preserving fashion. We implement the additive secret
sharings split of the secrets, the circuit parsing, the corresponding gates, the
Beaver triplets generation, update the network operations and describe our own
complex circuit assuming the presence of a trusted third party during the two
following weeks. Two more weeks to adapt our system so that it is able to
generate Beaver triplets with no trusted third party but using BFV homomorphic
encryption handled by Lattigo. Finally, we dedicate one week to revise our
implementations, compare and evaluate them.

Give a brief introduction about the aim of the project, and your road-map about the design/implementation.
\section{Part I}

\subsection{Threat model}
Give the corresponding threat model for the first part of the project that you implemented. 
\subsection{Implementation details}
\begin{itemize}
    \item Give your implementation details
    \item Detail the circuit you created at the end of the first part
\end{itemize}
\section{Part II}
\subsection{Threat model}
Give the corresponding threat model for the second part of the project that you implemented. 
\subsection{Implementation details}
Give implementation details.
\section{Evaluation}
- Give a comprehensive comparison and evaluation about Part1 and Part2 of the project including performance results. Feel free to use charts, tables, plots...\\
\begin{itemize}
    \item What affects the efficiency of the executions? Be specific, which types of operations/circuits are directly linked to performance?
    \item Is there any difference in terms of performance between Part I and Part II? Why? 
\end{itemize}

\section{Discussion}
\begin{itemize}
    \item Comment on your findings, discuss different outcomes for each part.
    \item Discuss outcomes from different circuits including your own circuit.
    \item In your opinion, which model is appropriate to use under which conditions/threat model? Why? Discuss.
    \item Come up with a scenario for each part of the implementation, discuss why it makes sense to use homomorphic encryption based generation of Beaver triplets.
\end{itemize}

\section{Conclusion}
\begin{itemize}
    \item Assess your learning outcomes for this project.
    \item What did you do? What did you learn? Any interesting design ideas? 
\end{itemize}

\bibliographystyle{IEEEtran}
\bibliography{bib}
\end{document}
